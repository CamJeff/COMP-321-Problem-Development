\documentclass[11pt]{article}
\usepackage{amsmath, amssymb}
\usepackage{hyperref}
\usepackage{graphicx}
\usepackage{xcolor}
\usepackage{array}
\usepackage{wrapfig}
\usepackage[margin=2.5cm]{geometry}
\usepackage[most]{tcolorbox}

% Define \problemname and \illustration if not using Kattis template
\newcommand{\problemname}[1]{\section*{#1}}
\newcommand{\illustration}[3]{%
    \begin{figure}[h]
        \centering
        \includegraphics[width=#1\textwidth]{#2}
        \caption{#3}
    \end{figure}
}

\tcbset{
  samplebox/.style={
    colback=gray!10,   % light gray background
    colframe=gray!50,  % subtle border
    boxrule=0.4pt,
    arc=2mm,           % rounded corners
    left=4mm,right=4mm,top=2mm,bottom=2mm,
  }
}

\begin{document}

\problemname{Procrastination Optimization}

% Example use of the \illustration command
% \illustration{0.3}{filename.jpg}{Photo by \href{url_here}{link text here, e.g. author name}}

\begin{wrapfigure}{r}{0.35\textwidth}
    \centering
    \vspace{-15pt} % optional: tweak vertical position
    \includegraphics[width=\linewidth]{Im.png}
    \vspace{5pt} % optional
\end{wrapfigure}

You want to write a program that will allow you to optimize the Kattis problems you 
select when completing your COMP 321 assignments. For each assignment, you only need 
to answer a subset of the listed questions to reach full marks. Each problem has a:

\begin{itemize}
    \item Point value (some positive integer)
    \item Difficulty rating (an integer from 1.0 to 10.0)
    \item Topic (like trees, graphs,...)
    \item Text length (number of words)
\end{itemize}

\noindent Although you’re a lazy student, you still care about your grades, and so you want 
to achieve full marks on your assignment while putting in the least amount of effort 
possible. On a given assignment, you need to attain at least $M$ points, and you want 
to do so while trying to:


\begin{enumerate}
    \item Minimize the total difficulty of the chosen problems first, then
    \item Minimize the number of problems solved. Then,
    \item Align the topics of the problems with concepts you have the best understanding of. Then finally,
    \item Minimize the total length of text read
\end{enumerate}

\noindent These requirements are listed in preferential order. That is, you first and foremost want to minimize the difficulty of questions you select. If there are still some choices for question subsets, then you can narrow it down further by minimizing the number of points solved, and work down to the least pressing requirement, which is minimizing the total length of text you read.


\section*{Input}

The first line contains two integers $M$ and $N$, which are the minimum total points needed for full score and the number of problems on an assignment, respectively. Note that $M$ can, and will be quite large, prepare for inputs where $M\geq 10^{20}$.

\bigskip

\noindent The next line contains a list of distinct strings: a variety of problem topics ranked from your best understanding to your worst understanding.

\bigskip

\noindent The following $N$ lines describes the problems on the assignment in the following format:

$$\texttt{Problem\_No.}\;\;\;\;\;\;\texttt{Points}\;\;\;\;\;\;\texttt{Difficulty}\;\;\;\;\;\;\texttt{Topic}\;\;\;\;\;\;\texttt{Text\_Length}$$


\noindent Where ``\texttt{Problem\_No}" $\in [1,N]$, is the problem number on the list of questions, ``\texttt{Points}"$\in[0,M]$ is the number of points that problem is worth,  ``\texttt{Difficulty}" $\in[1.0,10.0]$ is the difficulty rating of the problem (larger rating means more difficult), ``\texttt{topic}" is a non-empty string without spaces (assume exactly one topic is assigned to each problem) which is the topic the problem is assigned to, and finally, $1\leq \texttt{Text\_Length} \leq 10^4$ is the number of words in the problem statement. 

\section*{Output}

One line, containing the problem numbers, separated by spaces, that you choose in order to complete your assignment under the above conditions. If no subset of problems can achieve at least $M$ points, output $-1$. 

\bigskip


\begin{center}
  % Sample Input 1
  \begin{minipage}[t]{0.48\linewidth}
    \textbf{Sample Input 1}\\[4pt]
    \begin{tcolorbox}[samplebox]
      \texttt{10 4\\
      dp graphs arrays\\
      1 5 3 dp 120\\
      2 6 5 graphs 200\\
      3 4 1 arrays 50\\
      4 8 4 dp 300}
    \end{tcolorbox}
  \end{minipage}
  \hfill
  % Sample Output 1
  \begin{minipage}[t]{0.48\linewidth}
    \textbf{Sample Output 1}\\[4pt]
    \begin{tcolorbox}[samplebox]
      \texttt{3 4}
    \end{tcolorbox}
  \end{minipage}
\end{center}


\bigskip


\begin{center}
  % Sample Input 2
  \begin{minipage}[t]{0.48\linewidth}
    \textbf{Sample Input 2}\\[4pt]
    \begin{tcolorbox}[samplebox]
      \texttt{10 3 \\
stacks queues trees \\
1 2 1 stacks 100 \\
2 2 1 stacks 100 \\
3 10 9 trees 100 \\
}
    \end{tcolorbox}
  \end{minipage}
  \hfill
  % Sample Output 2
  \begin{minipage}[t]{0.48\linewidth}
    \textbf{Sample Output 2}\\[4pt]
    \begin{tcolorbox}[samplebox]
      \texttt{3}
    \end{tcolorbox}
  \end{minipage}
\end{center}

\bigskip


\begin{center}
  % Sample Input 3
  \begin{minipage}[t]{0.48\linewidth}
    \textbf{Sample Input 3}\\[4pt]
    \begin{tcolorbox}[samplebox]
      \texttt{100000000000000 4 \\
greedy dijkstra strings \\
1 60000000000000 3 greedy 5000 \\
2 50000000000000 2 dijkstra 8000 \\
3 40000000000000 1 strings 2000 \\
4 70000000000000 5 greedy 10000
}
    \end{tcolorbox}
  \end{minipage}
  \hfill
  % Sample Output 3
  \begin{minipage}[t]{0.48\linewidth}
    \textbf{Sample Output 3}\\[4pt]
    \begin{tcolorbox}[samplebox]
      \texttt{1 3}
    \end{tcolorbox}
  \end{minipage}
\end{center}





\end{document}
